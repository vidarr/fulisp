\documentclass[12pt]{article}
\usepackage{amsmath}
\usepackage{listings}
\lstset{
    stepnumber=1,
    language=Lisp
}

\title{\LaTeX}
\date{}
\begin{document}

% Remember: Use Oxford English!

\section{Introduction}

My intention of writing fuLisp is to figure out how stuff works.
The code should not be restricted to any platform if it is feasible to achieve.

\subsection{Principles}

\subsubsection{Independence}

One important principle of developing software is not inventing the wheel over
and over and over but design modularly and reuse these modules if you need 
their functionality.
The intention of writing fuLisp is to figure out how stuff works, to learn.
You do not learn by just copying code that already does what you want to learn
thus we are going to violate this principle in here for the sake of a higher 
goal.
Instead we will focus on \emph{independence}.
The code should not depend on anything external, or at least as as less
external functionality as possible.
This has a simple reason: You wont learn how stuff works by just telling it to 
work. Just calling a function \texttt{garbage\_collect()} will not unveil to you
how garbage collection is done.
Another point is portability.
Not all platforms, compilers and environments offer the same tools and 
functionality. 
This means to use as few foreign libraries. It also means to use the least 
common denominator of the language used to implement. For C, therefore we try 
to stick to ANSI C / C89 standard, even it prevents the use of really helpful 
functions like \texttt{snprintf(3)}.
Sticking to standards is a good idea anyhow if you want to maintain independence
of the environment your code will be compiled / used in.
At some point, you got to decide for one option, like the kind of operating 
system API you want to supprort. 
Using cross libraries could be one solution, but will come at the cost of 
becoming independent of the cross library.
Thus we decided to stick to Posix for example.

\subsubsection{\label{sec:simplicity}Simplicity}

Simplicity often competes with performance. 
However, our first goal is to learn, and obfuscating code by optimizing,
apart from wasting time on unimportant things, will make it harder to understand 
what is actually going on. 
Often algorithms are not that easy to understand in the first place, why 
further complicating it by scrambling the code?
Moreover, optimization is often done the wrong way optimizing stuff that 
does in fact not account much for time or memory consumption. 
There hides another rather important principle: First get it right, then get it 
fast.  
First ensure the correctness of the code, then if - and only if! - you encounter
performance problems, identify the bottlenecks and optimize them away!
So we decided to just leave it (mostly) entirely at first. 
We will apply certain optimizations if apropriate, but only after we got the 
code working and are able to ensure that the costs in terms of code complexity 
will be bearable.
Simple and thus more easily understandable code will tend to contain 
less bugs than complicated messy code, so adhering to simplicity is always 
recommended to stick to even on code not intended for educational purposes.
In fact, this principle turned out to be that important and successful 
throughout the history of software development that its well known as the 
\emph{KISS}-principle.
However, things are not as clear as they look at the first glance.
Consider designing a library, should you design for simple implementation or 
a simple interface?
if you design for a simple interface, it will usually induce a complicated 
internal structure. 
Consider the Posix Sockets or Java Streams.
There is one \texttt{read(2)} function in the Posix socket interface to read 
from files as well as from network connections or even memory.
Alternatively, there could be one function \texttt{read\_from\_file}, one 
function \texttt{read\_from\_tcp} and so on.
Of course, having one function for one single purpose would simplify the
implementation of each function, after all reading from a TCP stream involves
fairly other operations than reading from a file on the local file system.
However, simplifying the implemenation by increasing the number of functions 
within the interface would complicate the programs actually using the interface.
Thus, simplifying the internal implementation of a code module often comes with
a cost.
Another argument could be performance. 
The ACE library for instance deals with handling network streams, mainly by
wrapping calls of the Posix Socket API.
It got, however, not one \texttt{read()} function, but about a dozen, each one 
accepting different parameters. Of course, the intention is to give the user 
a maximum of flexibility to craft code suiting exactly the problem he addresses.
gaining the maximum of performance by tweaking all kinds of parameters.
Unfortunately, to even get an overview of what is actually offered and how to
work with these functions, figuring out which version suits best into the 
current frame takes ages and to use it properly you would probably have to spend
half a year training yourself.
What makes things even worse, the more complicated the interface is, the
more effort one has to spend to proper document all of it. 
Documenting one function \texttt{read()} is far less work than documenting a
dozen functions basically all doing the same, but, slightly different. 
It is probably not surprising that ACE being a great and performant library
nevertheless failed in respects of documenting itself. 
The more complex a library is, the more responsibility you transfer to the user
and the less likely it is, that the user will chose your module to use in the
first place.
All in all, we are convinced that it is better to simplify the interface 
than the internals.

\subsubsection{Modularity}

Yet another principle we will try to stick to as strictly as possible is 
modularity.
This means in order to solve a problem, split it up into smaller, fairly 
indepentend parts. 
This enables keeping things local and thus decreasing the complexity of a 
single part as smaller parts are easier to understand and oversee.
Working on a piece of software consisting of 500 lines of code is easier than 
working on 10000 lines of code at once.
Separating the problem into smaller independent parts reduces the amount of 
flaws in each part and promotes simplicity.
However, in the end the single bits and pieces need to fit together again and 
therefore their boundaries, their interface with their environment needs to 
be concise and well documented.
Therefore, we will focus on separating the software into moduels and make them 
as independent from each other as possible.
We try to designing clean, small and concise, easy to handle and well documented
interfaces.  
The internal implementation of the modules should be well done as well, of 
course, but priority has the interface definition and its documentation.
As everybody has only limited amount of resources, we will assume that whoever
takes a look into one specific module will want to become familiar with it 
anyhow thus investing some time we will not stress documenting the internals 
of a module but spend our power into crafting neat clean and well documented 
interfaces.
Besides, sinces each module is rather independent, it will not be to hard to
understand anyhow.
This fits, by coincidence, what we said in chapter \ref{sec:simplicity}.

\subsubsection{The Code is the Documentation}

%name vars, functions etc. 

% TODO: Incorporate
% \begin{itemize}
%     \item Adhere to standards: Use Standard C, POSIX
%     \item Try to provide standardized tests for each bit 
%     \item Simple design. Optimize only if benefits can be proven.
%     \item (Kind of follow-up to previous) Use lean interfaces (keep the headers
%         small)
% \end{itemize}


\section{Language Core}

\subsection{Reference}

The very core of the language that is actually dealt with by the \texttt{eval}
function consists of the functions shown in table \ref{tbl:CoreFunctions}.

\begin{table}[h]
    \centering
    \begin{tabular}{c|l}
        \texttt{QUOTE} & Prevents its argument from being evaluated \\
        \texttt{CONS} & \\
        \texttt{CAR} & \\
        \texttt{CDR} & \\
        \texttt{COND} & \\
        \texttt{CONS?} & \\
        \texttt{DEFINE} & Expects exactly two arguments. First argument is a
        symvol, second argument is any kind of expression. Creates a binding in
        the global environment, returns the expression bound.\\
        \texttt{BEGIN} & Expects a list of expressions. Evaluates each of these
        expressions in the order of their appearance. Returns the value of the last
        expression. \\
        \texttt{SET!} & \\
        \texttt{=} & \\
        \texttt{<} & \\
        \texttt{OR} & \\
        \texttt{AND} & \\
        \texttt{NOT} & \\
        \texttt{LAMBDA} & Creates a new unnamed closure. Expects exactly two
        arguments: First argument is a list of symbols, the second argument is a
        expression. If the returned closure is evaluated, these symbols will be
        bound to the arguments the closure is called with. The closure can be
        bound to a symbol and thus become a function as it is common.\\
        \texttt{+} & \\
        \texttt{*} & 
    \end{tabular}
    \caption{\label{tbl:CoreFunctions}}
\end{table}


\section{Basic Data Structures}

\subsection{Linked Lists}

\subsection{Last In - First Out - The Stack}

\subsection{Dictionaries - The Hash Table}


\section{Symbols}

\subsection{\texttt{NIL}}

\texttt{NIL} is general identified with the NULL-pointer. Whenever a
NULL-Pointer is encountered, it is interpreted to represent \texttt{NIL}.


\section{\texttt{lambda}}

\texttt{lambda} is represented by a structure that contains 

\begin{enumerate}
\item A list of symbols. These symbols will be inserted into the current symbol
table with the parameters of the lambda call as values. 
\item A pointer to a Block of code to be executed when being called.
\end{enumerate}

Whenever a lambda expression is called with parameters, the following steps will
be performed:

\begin{enumerate}
\item  the parameters' values are inserted into the current lookup table  with the 
symbols of the lambda function. If the symbols already exist within the table,
the symbols original values have to be saved.
\item The lambda values code block is executed.
\item The symbols original values are restored.
\end{enumerate}


\section{Symbol Tables}

\begin{enumerate}
\item There is one global Symbol table. This table contains \texttt{NIL} etc.
\item Other symbol tables can be created on demand, for example by package.
These symbol tables are inserted into the current symbol table.
\item If a package is entered, the current package is pushed onto a stack and
the symbol table of the package becomes the new one.
\end{enumerate}

Lookups are performed via

\begin{enumerate}
\item the symbol is looked up in the current lookup table.
\item If not found, the symbol table stack is searched for.
\item A name like \texttt{part1::part2::part3...} is split up into its parts. Then,
\texttt{part1} is looked up just as a normal symbol. If it does not resolve to a symbol
table, an error is produced. If it resolves to a symbol table, the other parts
\texttt{part2}, \texttt{part3} etc are looked up within this lookup table.
\end{enumerate}


\section{Input/Output}

Uttmost principle should be that if a expression is printed out, its textual
representation, fed back into the reader, should yield a equivalent expression,
that is the following relation should hold:

({\texttt READ} ({\texttt COERCE} {\texttt 'STRING} {\texttt EXPRESSION})) $\equiv$ {\texttt EXPRESSION}

\section{Implementation}

\subsection{Garbage Collection}

\subsection{Strings}

\lstset{language=C}  

\subsection{Introduction}


\subsection{The problem with strings in C}

In C, strings are an array of characters. Since in C there is no way of 
determining the actual number of elements in the array, in C the end of a 
string is marked by a \texttt{'\textbackslash0'} - character.
All functions for manipulating strings will rely on the last character of a
string is a \texttt{'\textbackslash0'}. 
Determining the length of a string is done by the standard library funtion
\texttt{strcpy(3)} , which could be implemented like 

\begin{lstlisting}
size_t strlen(char *str) {
    char *current = str;
    while(*current != '\0') current++;
    return current - str;
}
\end{lstlisting}

For formatting strings, there is \texttt{sprintf(3)} .
It is used like

\begin{lstlisting}
#define MAX_BUF_LEN 20
char *destString = (char *)malloc(sizeof(char) * (MAX_BUF_LEN + 1));
sprintf(destString, "In Code line %u\n", __LINE__);
\end{lstlisting}

In here, a severe problem arises: What happens if the formatted result exceeds
the allocated length of 20 characters?
At first, \texttt{sprintf(3)} itself will write beyond the end of the allocated
 memory, which is a serious problem: 
Data following the allocated space will be overwritten. 
This is bad already, but the subsequent consequences are potentially much worse.

Consider the following code:

\begin{lstlisting}
char *source = (char *)malloc(sizeof(har) * 2);
char *dest   = (char *)malloc(sizeof(har) * 2);
source[0] = 'a';
source[1] = 'b'; /* Should be '\0' */
strcpy(dest, source);
\end{lstlisting}

\texttt{strcpy(3)} expects that source terminates with '\textbackslash0' and 
will copy bytes from source to dest until it encounters 
\texttt{'\textbackslash0'}.
You see easily that in the example above source does not terminate with a zero.
Since \texttt{strcpy(3)} has no means to determine the actual length of neither 
the size of the memory allocated for source nor for dest it will overwrite data 
behind the end of dest until it encounters a \texttt{'\textbackslash0'} behind 
source by accident.
This could overwrite thousands of bytes in memory and corrupt it.
If you are lucky, this will result in a segmentation fault and thus tell you 
that there is a problem. 
If you are not quite as lucky the error will remain unnoticed initially, but
make your program crash at a random point in the program flow leaving the
programmer with no hint towards the actual error causing the crash.

This leaves us with two rules that should be followed when dealing with strings:

\begin{enumerate}
\item Try to avoid buffer overflows.
\item If you cannot avoid overflows, at least reveal them.
\end{enumerate}

\subsection{Mitigation}

How to reach these goals?
For one thing, always remember the length of your buffers.
For instance, recent C versions provide \texttt{snprintf(3)} as an replacement 
for \texttt{sprintf(3)}. \texttt{snprintf(3)} takes an additional size argument
 and write at most size characters into the buffer and if it would happen, 
notify the caller about it:

\begin{lstlisting}
#define MAX_BUF_LEN 20
size_t lengthOfTargetString = 0;
char *destString = (char *)malloc(sizeof(char) * (MAX_BUF_LEN + 1));
lengthOfTargetString = snprintf(destString, MAX_BUF_LEN, 
    "In Code line %u\n", __LINE__);
if(lengthOfTargetString > MAX_BUF_LEN) {
    fprintf(stderr, "String could not be created\n");
}
\end{lstlisting}

Of course, it is a littlebit clumsy to always having to carry around two 
values for each string - the string itself and its length. One solution could be
to wrap both within a struct type and completely refraining from passing around
bare char - pointers:

\begin{lstlisting}
struct String {
    char *data;
    size_t maxLength;
};
\end{lstlisting}

This induces some overhead, but would be an acceptable solution. 
Always trying to optimize each an everything, one could have the idea to
use a more compact design like just appending the length directly in memory
after the char pointer like this:

\begin{lstlisting}
size_t strLength = 5;
char *string = (char *)malloc(sizeof(char) * strLength + sizeof(size_t));
string[0] = 'a';
string[1] = 'b';
string[3] = '\0';
*((size_t *)(string + strLength)) = strLength;
\end{lstlisting}

However, this is a bad approach since if somewhere in your code something
overwrites the \texttt{'\textbackslash0'} and beyond, it will overwrite the 
length of the string buffer.
Another solution could be to prepending it to the string:
 
\begin{lstlisting}
size_t strLength = 5;
char *string = (char *)malloc(sizeof(char) * strLength + sizeof(size_t));
((size_t *)string)[0] = strLength;
string = (char *)(((size_t *)string) + 1);
string[0] = 'a';
string[1] = 'b';
string[3] = '\0';
\end{lstlisting}

However, this makes it very diffcult to deal with. Which pointer will you store?
If you use the pointer pointing at the length field, you will have to add 
a displacement whenever you want to access the char array representing the
actual string. Clumsy and error-prone.
If you store the pointer after the length field, you will be easily accessing 
the actual string data. You could use the pointer like it was an ordinary 
char array and pasing it to whichever C string function. 
For accessing the actual length field, you have to decrease the pointer 
appropriately which is not too bad since you could wrap this operation in 
appropriate functions.
However, consider the following piece of code:

\begin{lstlisting}
size_t lengthOfTargetString;
size_t strLength = 20;
char *string = (char *)malloc(sizeof(char) * strLength + sizeof(size_t));
((size_t *)string)[0] = strLength;
string = (char *)(((size_t *)string) + 1);
lengthOfTargetString = snprintf(destString, getMaxBufferLength(string),
    "In Code line %u\n", __LINE__);
if(lengthOfTargetString > MAX_BUF_LEN) {
    fprintf(stderr, "String could not be created\n");
    string[get_max_buf_length(string)] = '\0';
}
printf("%s\n", string);
free(string);  /*  This will cause trouble */
\end{lstlisting}

The \texttt{free(3)} call will fail, since string does not point to the 
beginning of an allocated block put straight into the middle of such a block.
Of course you could introduce a function \texttt{getStartOfBuffer(string)}
which decreases the pointer string appropriately to point to the correct start.
However, this is again error-prone since relying on the user to remember
to care is never a good idea. Somebody will forget.

Thus the only consequency one can conclude from this is to always remember 
the length of a buffer but keep the length and the actual buffer separate.

/subsection{Detecting Buffer Overflows}

So everything is fine? Not quite. Not all string functions expect length
arguments. \texttt{snprintf(3)} for example has been introduced with C99 only, 
thus if you develop for a compiler only supporting C89, you will have to either
reimplement \texttt{snprintf(3)} on your own or fall back to 
\texttt{sprintf(3)}.
Now, since \texttt{sprintf(3)} does not care for the length of the provided 
buffer, there is no way to avoid a buffer overflow.
But the least we can do is ensuring that such a incident will not remain hidden.
The solution to this are canaries.

Consider this:

\begin{lstlisting}
char *strBuffer = (char *)malloc(sizeof(char) * (maxBufferLength + 1));
strBuffer[maxBufferLength] = '\0';
sprintf(strBuffer, "In line %u\n", __LINE__);
if(strBuffer[maxBufferLength] != '\0') {
    fprintf(stderr, "ERROR: Buffer overflow occured!\n");
}
\end{lstlisting}

In here, we mark the end of the buffer with a zero char.
If sprintf writes less than the maximum amount of chars to \texttt{strBuffer}, 
nobody will care for the zero char beyond the zero char marking the end of 
the string.
If it writes exactly \texttt{maxBufferLength} it will overwrite 'our' zero char 
with its own zero char terminating the written string. The buffer will still 
end with a zero char.
If \texttt{sprintf(3)} writes more than \texttt{maxBufferLength} chars, it will 
overwrite the zero char at the end of the buffer and possibly beyond the end of
the buffer.
This is bad, but now one can figure out that a buffer overflow has occured 
by checking the very last char in the buffer. 
If it is a zero char, everything is fine.
If it is not a zero char, a buffer overflow has occured and memory 
corrupted.

If a string function encounters that a string has been tainted, what should it 
do about it?
One reasonable reaction would be to just pass it back to the caller and let
him take care of it all.
However, you must take into account that the caller will not always check for
the string being tainted and just go on working with it and manipulating it by
passing it for example to \texttt{strcpy(3)}.
We could avoid the problems arising from this behavior by just placing 
the zero terminator again at the very end of the string, however this would mask
the buffer overflow error again.

We can do better than that by keeping the last character in the string to 
be something diferent from \texttt{'\textbackslash0'} but setting the character
before the last character to \texttt{'\textbackslash0'}:

\begin{lstlisting}
int    requiredBufLen;
size_t bufLen = 20;
char *string = (char *)malloc(sizeof(char) * (bufLen + 1));
string[bufLen] = '\0';
requiredBufLen = sprintf(string, bufLen, 
   "We are in line %u\n", __LINE__);
if(requiredBufLen >= bufLen || string[bufLen] != '\0')  {
   fprintf(stderr, "String has been tainted!\n");
   string[bufLen - 1] = '\0';
}
\end{lstlisting}

Now, this block ensures that
\begin{enumerate}
\item A buffer overflow within \texttt{sprintf(3)} wont go unnoticed since you 
still can check whether there has been an buffer overflow through 
\texttt{string[bufLen] == '\textbackslash0'}
\item The string will stay marked as having been tainted.
\item Subsequent buffer overflows stemming handling the string naively 
will not lead to subsequent buffer overflows. 
\end{enumerate}

If the validity of the string is crucial and the code should fail as soon as 
possible, you can just terminate the program instead of printing to stderr in 
the example above. Otherwise if it is not crucial then this code ensures that 
there will no subsequent buffer overflows in the aftermath.

\lstset{language=Lisp}  


\section{Optimisations}

\subsection{Memory Preallocation}

\subsubsection{Benchmark}


Compiled the REPL twice, once with memory preallocation enabled, once with
memory preallocation disabled.
Then started both REPLs in parallel, and evaluated one form subsequently in the
one REPL, then the other, repeated this several times.
The form to be evaluated is

\begin{verbatim}
    (CONS (CONS (CONS (CONS (CONS 
      (CONS (CONS (CONS (CONS (CONS 
        (CONS (CONS (CONS (CONS (CONS 
          (CONS (CONS (CONS (CONS (CONS 
            (CONS (CONS 0 1) 2)
          3) 4) 5) 6) 7) 8) 9) 10) 11) 12) 
        13) 14) 15) 16) 17) 
      18) 19) 20) 21) 22)
\end{verbatim}

Results are shown in table \ref{tbl:BMResultsMemPreallocation}.
As you can see, for this example we gain about 9\% of speed.
We will not ignore the fact that this example has been tailored for making
abundand memory requests.

\begin{table}
    \centering
    \begin{tabular}{r|c|c}
           & With  & Without \\
        \hline 
         1 & 22038 & 24153 \\
         2 & 22157 & 24841 \\
         3 & 22204 & 24377 \\
         4 & 22238 & 25251 \\
         5 & 22315 & 24141 \\
         6 & 22815 & 24792 \\
         7 & 22146 & 24134 \\
         8 & 22268 & 24498 \\
         9 & 21969 & 24489 \\
        10 & 21720 & 24750 \\
        \hline
        summary & 221870 & 245426 \\
        \hline
        average & 22187  & 24543
    \end{tabular}
    \caption{\label{tbl:BMResultsMemPreallocation} Ticks measured for evaluating
        the form given in the text, left with memory preallocation enabled, on
    the right disabled.}
\end{table}


\subsection{Packed Expression Structure}

\subsubsection{Benchmark}


\begin{table}
    \centering
    \begin{tabular}{r|c|c}
           & flat  & packed \\
        \hline 
         1 & 358995 & 714173 \\
         2 & 506777 & 360570 \\
         3 & 734729 & 444427 \\
         4 & 719196 & 664277 \\
         5 & 730034 & 401226 \\
         6 & 592128 & 460916 \\
         7 & 673055 & 383550 \\
         8 & 724243 & 741072 \\
         9 & 486136 & 361164 \\
        10 & 728938 & 732850 \\
        \hline
        summary & 6254231 & 732850 \\
        \hline
        average & & 
    \end{tabular}
    \caption{\label{tbl:BMResultsExpressionFormat} Ticks measured for performing
        test-eval, left with the initial, flat structure, on
    the right with the packed format.}
\end{table}
\end{document}
